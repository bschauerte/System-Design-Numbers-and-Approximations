%%%%%%%%%%%%%%%%%%%%%%%%%%%%%%%%%%%%%%%%%
% System Design Numbers and Approximations Cheat Sheet
%
% Boris Schauerte
% Version 1.o (4/16/2022)
%
% Sources
% - https://matthewdbill.medium.com/back-of-envelope-calculations-cheat-sheet-d6758d276b05
% - https://gist.github.com/jboner/2841832
%.  - By Jeff Dean:               http://research.google.com/people/jeff/
%.  - Originally by Peter Norvig: http://norvig.com/21-days.html#answers
%
% License:
% The MIT License
%
%%%%%%%%%%%%%%%%%%%%%%%%%%%%%%%%%%%%%%%%%

%%%%%%%%%%%%%%%%%%%%%%%%%%%%%%%%%%%%%%%%%
% Cheatsheet
% LaTeX Template
% Version 1.0 (12/12/15)
%
% This template has been downloaded from:
% http://www.LaTeXTemplates.com
%
% Original author:
% Michael Müller (https://github.com/cmichi/latex-template-collection) with
% extensive modifications by Vel (vel@LaTeXTemplates.com)
%
% License:
% The MIT License (see included LICENSE file)
%
%%%%%%%%%%%%%%%%%%%%%%%%%%%%%%%%%%%%%%%%%

%----------------------------------------------------------------------------------------
%	PACKAGES AND OTHER DOCUMENT CONFIGURATIONS
%----------------------------------------------------------------------------------------

\documentclass[11pt]{scrartcl} % 11pt font size

\usepackage[utf8]{inputenc} % Required for inputting international characters
\usepackage[T1]{fontenc} % Output font encoding for international characters

\usepackage[margin=0pt, landscape]{geometry} % Page margins and orientation

\usepackage{graphicx} % Required for including images

\usepackage{color} % Required for color customization
\definecolor{mygray}{gray}{.75} % Custom color

\usepackage{url} % Required for the \url command to easily display URLs

\usepackage[ % This block contains information used to annotate the PDF
colorlinks=false, 
pdftitle={System Design Cheatsheet}, 
pdfauthor={Boris Schauerte}, 
pdfsubject={Numbers and Approximations for System Design (Interviews)}, 
pdfkeywords={System Design, Computer Science, Back of the Envelope}
]{hyperref}

\usepackage{amsmath}

\setlength{\unitlength}{1mm} % Set the length that numerical units are measured in
\setlength{\parindent}{0pt} % Stop paragraph indentation

\renewcommand{\dots}{\ \dotfill{}\ } % Fills in the right amount of dots

\newcommand{\command}[2]{#1~\dotfill{}~#2\\} % Custom command for adding a shorcut

\newcommand{\sectiontitle}[1]{\paragraph{#1} \ \\} % Custom command for subsection titles

%----------------------------------------------------------------------------------------

\begin{document}

\begin{picture}(297,210) % Create a container for the page content

%----------------------------------------------------------------------------------------
%	TITLE SECTION 
%----------------------------------------------------------------------------------------

\put(10,200){ % Position on the page to put the title
\begin{minipage}[t]{210mm} % The size and alignment of the title
\section*{System Design Cheatsheet - Numbers and Approximations - v1} % Title
\end{minipage}
}

%----------------------------------------------------------------------------------------
%	FIRST COLUMN SPECIFICATION∫√∫
%----------------------------------------------------------------------------------------

\put(10,180){ % Divide the page
\begin{minipage}[t]{85mm} % Create a box to house text

%----------------------------------------------------------------------------------------
%	HEADING ONE
%----------------------------------------------------------------------------------------

\sectiontitle{Users to Volume}

\command{x Million users * y KB}{xy GB}
\command{x Million users * y MB}{xy TB}

\sectiontitle{Period Numbers}

\begin{center}
\begin{tabular}{ l r r r }
per Month & 1 Billion & 1 Million & 1 Thousand \\ 
%per Month & 1 G & 1 M & 1 K \\ 
per Day & 32 M & 32 K & 32 \\  
per Hour & 1.3 M & 1.3 K & 1.3 \\
per Minute & 22 K & 22 & 0.02 \\
per Second & 400 & 0.4 & 0.0004   
\end{tabular}
\end{center}

\begin{center}
\begin{tabular}{ l r r r }
per Day & 1 Billion & 1 Million & 1 Thousand \\ 
%per Day & 1 G & 1 M & 1 K \\ 
per Hour & 42 M & 42 K & 42 \\  
per Minute & 700 K & 700 & 0.7 \\
per Second & 12 K & 12 & 0.01
\end{tabular}
\end{center}

Example 1: If a server has a million requests per day, it will need to handle 12 requests per second.\\

Example 2: 100M photos (200KB) are uploaded daily to a server. 100 (number of millions) * 12 (the number per second for 1M) = 1200 uploads a second. 1200 (uploads) * 200KB (size of photo) = 240MB per second.

\sectiontitle{Number Sizes}

\command{\textbf{K}ilo}{Thousands (3 zeros)}
\command{\textbf{M}ega}{Millions (6 zeros)}
\command{\textbf{G}iga}{Billions (9 zeros)}
\command{\textbf{T}era}{Trillions (12 zeros)}
\command{\textbf{P}eta}{Quadrilions (15 zeros)}

%\command{1 Byte = 8 Bits}
%\command{1 KB = 1000 B}
%\command{1 MB = 1000 KB}

%\sectiontitle{Lengths}
%
%\command{Max. URL size}{approx. 2000 digits}
%\command{ASCII charset}{128}
%\command{Unicode charset}{143,859}

%----------------------------------------------------------------------------------------

\end{minipage} % End the first column of text
} % End the first division of the page

%----------------------------------------------------------------------------------------
%	SECOND COLUMN SPECIFICATION 
%----------------------------------------------------------------------------------------

\put(105,180){ % Divide the page
\begin{minipage}[t]{85mm} % Create a box to house text

%----------------------------------------------------------------------------------------
%	HEADING FOUR
%----------------------------------------------------------------------------------------

%----------------------------------------------------------------------------------------
%	HEADING TWO
%----------------------------------------------------------------------------------------				
			
\sectiontitle{Service Limitations}

These are very rough estimations on throughput, requests, and connections (Conn.) that certain services can handle.\\

% see https://matthewdbill.medium.com/back-of-envelope-calculations-cheat-sheet-d6758d276b05

\begin{center}
\begin{tabular}{ l r r r }
-              & Storage & Conn. & Requests \\ 
SQL DB         & 60 TB & 30 K & 25 K/sec \\  
Cache (Redis). & 300 GB & 10 K & 100 K/sec \\
\end{tabular}
\end{center}

\begin{center}
\begin{tabular}{ l r r r }
-              & Throughput & Requests \\ 
Web Server     & - & 5-10 K/sec \\
Queues/Streams & 1-100 MB/s & 1-3 K/sec % 1MB-50MB/s (Write) / 2MB-100MB/s (Read)
\end{tabular}
\end{center}
					
%----------------------------------------------------------------------------------------
%	HEADING FIVE
%----------------------------------------------------------------------------------------				
					
\sectiontitle{Throughput} % Heading five

\command{Read sequentially from memory}{4 GB/s}
\command{Read sequentially from SSD}{1 GB/s}
\command{Read sequentially from HDD}{30 MB/s}
\command{Read sequentially from 1Gbps Ethernet}{100 MB/s}

\sectiontitle{Latency} % Heading five

\command{Read 1 MB sequentially from memory}{0.25 ms}
\command{Read 1 MB sequentially from SSD}{1 ms}
\command{Read 1 MB sequentially from HDD}{20 ms}
%\command{Cross continental network}{6–7 world-wide round trips per second.}

%\command{Roundtrip within datacenter}{0.5 ms (500 \(\mu\)s)}
\command{Roundtrip within datacenter}{0.5 ms (500 us)}
\command{Send packet CA $\rightarrow$ NL $\rightarrow$ CA}{150 ms}

%\command{L1 cache reference}{0.5 ns}
%\command{L1 cache reference}{0.5 ns}
%\command{Branch mispredict}{5 ns}
%\command{L2 cache reference}{7 ns}
%\command{Mutex lock/unlock}{25 ns}

%L1 cache reference                           0.5 ns
%Branch mispredict                            5   ns
%L2 cache reference                           7   ns                      14x L1 cache
%Mutex lock/unlock                           25   ns
%Main memory reference                      100   ns                      20x L2 cache, 200x L1 cache
%Compress 1K bytes with Zippy             3,000   ns        3 us
%Send 1K bytes over 1 Gbps network       10,000   ns       10 us
%Read 4K randomly from SSD*             150,000   ns      150 us          ~1GB/sec SSD
%Read 1 MB sequentially from memory     250,000   ns      250 us
%Round trip within same datacenter      500,000   ns      500 us
%Read 1 MB sequentially from SSD*     1,000,000   ns    1,000 us    1 ms  ~1GB/sec SSD, 4X memory
%Disk seek                           10,000,000   ns   10,000 us   10 ms  20x datacenter roundtrip
%Read 1 MB sequentially from disk    20,000,000   ns   20,000 us   20 ms  80x memory, 20X SSD
%Send packet CA->Netherlands->CA    150,000,000   ns  150,000 us  150 ms
%
%Notes
%-----
%1 ns = 10^-9 seconds
%1 us = 10^-6 seconds = 1,000 ns
%1 ms = 10^-3 seconds = 1,000 us = 1,000,000 ns

%----------------------------------------------------------------------------------------

\end{minipage} % End the second column of text
} % End the second division of the page

%----------------------------------------------------------------------------------------
%	THIRD COLUMN SPECIFICATION 
%----------------------------------------------------------------------------------------

\put(200,180){ % Divide the page
\begin{minipage}[t]{85mm} % Create a box to house tex

%----------------------------------------------------------------------------------------
%	IMPORTANT FILES
%----------------------------------------------------------------------------------------

\sectiontitle{Data Sizes}
			
\command{char}{1 Byte (8 Bit}
\command{char (Unicode)}{2 Byte (16 Bit}
\command{short}{2 Byte (16 Bit)}
\command{int or float}{4 Byte (32 Bit}
\command{long or double}{8 Byte (64 Bit}

\sectiontitle{Approximate Object Sizes}

\command{File}{100 KB}
\command{Web Page w/o a lot of magic and images}{100 KB}
\command{Picture (jpeg, ...)}{200 KB}
\command{Short Posted Video}{2 MB}
\command{Streaming Video}{50 MB/s}

%----------------------------------------------------------------------------------------
%	FOOTNOTE
%----------------------------------------------------------------------------------------

\vspace{\baselineskip}
\linethickness{0.5mm} % Thickness of the footer line
{\color{mygray}\line(1,0){30}} % Print the line with a custom color

\footnotesize{
Created by Boris Schauerte, 2022\\ 
				
Released under the MIT license.
}

%----------------------------------------------------------------------------------------

\end{minipage} % End the third column of text
} % End the third division of the page
\end{picture} % End the container for the entire page

%----------------------------------------------------------------------------------------

\end{document}